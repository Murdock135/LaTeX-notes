% \textbf{"Given some input data, output data, and a feasible input space,
% determine the optimal combination of input data to achieve the desired
% output"}—this is the fundamental problem we seek to address. For instance,
% consider a scenario where the per-unit profit for Product A and Product B is
% unknown, but historical data on production quantities and corresponding total
% profits is available. The task is to determine the optimal production quantities
% of Product A and Product B to maximize profit, given resource constraints,
% without explicit knowledge of per-unit profits.

% Traditional data-driven techniques, such as Machine Learning (ML) and
% statistical methods, have achieved remarkable success in prediction and
% inference. However, they are often inadequate for problems where the goal is not
% only to predict but to prescribe decisions that optimize outcomes under
% constraints. For instance, while ML models can predict profits based on past
% data, they do not inherently account for how the predicted outcomes influence or
% are influenced by decision variables.

% This paper argues that solving such problems requires integrating learning with
% optimization to form a cohesive decision-making framework. Specifically, we
% propose a framework rooted in \textbf{Predict, then Optimize}, which aims to
% bridge the gap between predictive models and optimization problems.

% The rest of the paper is structured as follows: Section 2.1 discusses the
% central tenets of machine learning and its limitations in solving
% decision-making problems. Section 2.2 explores existing techniques for
% decision-making using data, such as optimization, stochastic programming, and
% data-driven control. Finally, we propose our framework in Section 3 and outline
% potential applications and future directions.


% \textbf{"Given some input data, output data and a feasible input space,
% determine the optimal combination of the input data for the optimal output
% data"}- Fundamentally, This is the problem we seek to solve. For example,
% Suppose we don't know the exact profit per unit for Product A and Product B, but
% we have historical data on production quantities and the corresponding total
% profits. Our goal is to determine the optimal production quantities of Product A
% and Product B to maximize profit, given resource constraints and without
% explicit knowledge of per-unit profits.

% We will argue in this paper that the above task requires techniques beyond
% commonly used data driven techniques such as Machine Learning and statistics and
% finally propose a simple framework that tries to solve such a problem. The paper
% is structured in the following manner: section 2.1 discusses the central tenets
% of machine learning and how it can help solve the above problem. Section 2.2
% Discusses techniques that currently exist for similar problems i.e. problems
% that involve decision making using data.

% The dawn of mankind did not only mark the beginnging of the human race. It
% also marked the beginning of a battle between this new organism, `mankind',
% and the previous residents of this planet- microorganisms. As we live and
% breathe, we blatantly ignore our cohabitants, bacteria and viruses. However,
% we are never free from their clutches. Every year, millions of humans undergo
% a period- may be as short as a day or as long as several years- of struggle,
% battling against the effects of bacteria such as E.coli (Escherichia coli),
% Salmonella, Streptococcus pyogenes, Staphylococcus aureus, Clostridium
% botulinum, etc. These bacteria thrive on certain ambient conditions and
% environments. These environments can be of varying characterisitcs. Some
% thrive in moist, cold environments and some in hot environments and then some
% others in different kinds of surfaces. For example, halophiles trive on high
% salt concentrations while Barophiles thrive on high pressure conditions. 

% Machine learning (ML) is a field that primarily focuses on predicting some
% outcome, given some data. The central tenet of using such a framework is that
% the input variables that are used are somehow correlated with the outcome. For
% example, in the popular iris dataset, the input variables are the sepal and
% petal characterisitcs and the outcome is one of three species of the iris
% flower. The secondary tenet is that no observation made in this world is by
% itself the true representation of what is being observed and so \textit{all
% observations} are noisy (contain noisy information). Thus, it follows that a
% single prediction made from a single noisy observation is also noisy (Noisy
% input gives noisy output). However, time and time again, such predictions have
% proven helpful. For example, a $\geq90\%$ accurate transformer is deemed capable
% enough to complete incomplete passages or poems, computer vision systems are
% used to identify objects, understand human sentiments, etc. These wonders are
% due to the fact that all phenomena are generated by probability distributions
% and a random sample approaches a limiting probability distribution as the size
% of the sample increases. Thus, even though the learner is unaware of the
% underlying limiting distribution, as we increase the sample size, the particular
% error-correction algorithm of the learner leads it towards the true prediction. 
