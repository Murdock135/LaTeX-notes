\documentclass[12pt, letterpaper]{article}
\usepackage{setspace}
\usepackage{subcaption}
\usepackage{graphicx}
\usepackage{array}
\usepackage{longtable}
\usepackage{quotes}
\usepackage{amsmath}
\usepackage{hyperref}
\usepackage{xcolor}

% set reference files
\usepackage{biblatex}
\addbibresource{references.bib}

% set margins
\usepackage{geometry}
\geometry{margin=1in}

% Remove paragraph indentation
\setlength{\parindent}{0pt}

\title{ECE 8725 Final Project\\
    \large Connecting Data Driven Optimization, Decision focused learning and Symbolic Regression}
\author{Qazi Zarif Ul Islam}

\onehalfspacing
\begin{document}
\maketitle

\section{Abstract}

\section{Introduction}

% \textbf{"Given some input data, output data, and a feasible input space,
% determine the optimal combination of input data to achieve the desired
% output"}—this is the fundamental problem we seek to address. For instance,
% consider a scenario where the per-unit profit for Product A and Product B is
% unknown, but historical data on production quantities and corresponding total
% profits is available. The task is to determine the optimal production quantities
% of Product A and Product B to maximize profit, given resource constraints,
% without explicit knowledge of per-unit profits.

% Traditional data-driven techniques, such as Machine Learning (ML) and
% statistical methods, have achieved remarkable success in prediction and
% inference. However, they are often inadequate for problems where the goal is not
% only to predict but to prescribe decisions that optimize outcomes under
% constraints. For instance, while ML models can predict profits based on past
% data, they do not inherently account for how the predicted outcomes influence or
% are influenced by decision variables.

% This paper argues that solving such problems requires integrating learning with
% optimization to form a cohesive decision-making framework. Specifically, we
% propose a framework rooted in \textbf{Predict, then Optimize}, which aims to
% bridge the gap between predictive models and optimization problems.

% The rest of the paper is structured as follows: Section 2.1 discusses the
% central tenets of machine learning and its limitations in solving
% decision-making problems. Section 2.2 explores existing techniques for
% decision-making using data, such as optimization, stochastic programming, and
% data-driven control. Finally, we propose our framework in Section 3 and outline
% potential applications and future directions.


% \textbf{"Given some input data, output data and a feasible input space,
% determine the optimal combination of the input data for the optimal output
% data"}- Fundamentally, This is the problem we seek to solve. For example,
% Suppose we don't know the exact profit per unit for Product A and Product B, but
% we have historical data on production quantities and the corresponding total
% profits. Our goal is to determine the optimal production quantities of Product A
% and Product B to maximize profit, given resource constraints and without
% explicit knowledge of per-unit profits.

% We will argue in this paper that the above task requires techniques beyond
% commonly used data driven techniques such as Machine Learning and statistics and
% finally propose a simple framework that tries to solve such a problem. The paper
% is structured in the following manner: section 2.1 discusses the central tenets
% of machine learning and how it can help solve the above problem. Section 2.2
% Discusses techniques that currently exist for similar problems i.e. problems
% that involve decision making using data.

% The dawn of mankind did not only mark the beginnging of the human race. It
% also marked the beginning of a battle between this new organism, `mankind',
% and the previous residents of this planet- microorganisms. As we live and
% breathe, we blatantly ignore our cohabitants, bacteria and viruses. However,
% we are never free from their clutches. Every year, millions of humans undergo
% a period- may be as short as a day or as long as several years- of struggle,
% battling against the effects of bacteria such as E.coli (Escherichia coli),
% Salmonella, Streptococcus pyogenes, Staphylococcus aureus, Clostridium
% botulinum, etc. These bacteria thrive on certain ambient conditions and
% environments. These environments can be of varying characterisitcs. Some
% thrive in moist, cold environments and some in hot environments and then some
% others in different kinds of surfaces. For example, halophiles trive on high
% salt concentrations while Barophiles thrive on high pressure conditions. 

% Machine learning (ML) is a field that primarily focuses on predicting some
% outcome, given some data. The central tenet of using such a framework is that
% the input variables that are used are somehow correlated with the outcome. For
% example, in the popular iris dataset, the input variables are the sepal and
% petal characterisitcs and the outcome is one of three species of the iris
% flower. The secondary tenet is that no observation made in this world is by
% itself the true representation of what is being observed and so \textit{all
% observations} are noisy (contain noisy information). Thus, it follows that a
% single prediction made from a single noisy observation is also noisy (Noisy
% input gives noisy output). However, time and time again, such predictions have
% proven helpful. For example, a $\geq90\%$ accurate transformer is deemed capable
% enough to complete incomplete passages or poems, computer vision systems are
% used to identify objects, understand human sentiments, etc. These wonders are
% due to the fact that all phenomena are generated by probability distributions
% and a random sample approaches a limiting probability distribution as the size
% of the sample increases. Thus, even though the learner is unaware of the
% underlying limiting distribution, as we increase the sample size, the particular
% error-correction algorithm of the learner leads it towards the true prediction. 

\textit{"Given some input data, output data, and a feasible input space, determine the optimal combination of input data to achieve the desired output"}—this is the fundamental problem we seek to address. 

For instance, consider a scenario where the per-unit profit for Product A and
Product B is unknown, but historical data on production quantities and
corresponding total profits is available. The task is to determine the optimal
production quantities of Product A and Product B to maximize profit, given
resource constraints, without explicit knowledge of per-unit profits.

The above problem is an adaptation of the conventional optimization problem
where a problem is provided in the form of a function, which is to be minimized
or maximized. Only this time, the functional form is not provided and instead,
we are provided \textit{observations} about the inputs and the resulting output.
Machine Learning (ML) and statistical methods have achieved remarkable success
in prediction and inference with such historical data. But predictive models do
not account for how the predictions affect downstream decisions or outcomes. On
the other hand, techniques that focus on permuting inputs to achieve certain
outcomes- henceforth called `Decision Focued techniques'- do not leverage
observations. Thus, we think it is important to bridge this gap by using
principles from both branches- Machine Learning and Decision Focused techniques
to tackle the problem at hand.

The rest of the paper is structured as follows. Section \ref{sec: Background} discusses
the background of research in this area. Section \ref{sec: formalization} formalizes the problem, discusses how to tackle
the problem with Decision Focused Learning and finally how to tackle it with the
new proposed framework, \textit{Express Symbolically, then Optimize}.


\section{Background}
\label{sec: Background}
ML based products have been used primarily to make tasks more efficient, not to
induce social policy and decision. We trust qualitative arguments when deciding
where to build a hospital or where to put more health care funding. One
explanation for this is that we can determine biases in an argument instantly
and doing the same for a learner is much harder as there are different kinds of
biases and a plethora of experimental methods to discern them \cite{bias}.
Another reason for the insuitability of ML techniques in this area is that the
predictions do not take into account \textit{its own effects} on downstream
decisions. But where we have not trusted predictions, we have trusted
optimization and statistical testing. \textbf{Optimization techniques} have been
heavily relied on when a deterministic answer is deemed to exist. Examples of
such problems include The Travelling Salesman problem, The Knapsack problem,
etc. Apart from \textbf{Stochastic programming}, a particular paradigm of
optimization method, these problems differ from Machine Learning problems in
that the problem is not solved using data but rather a fixed set of conditions.
This fixed set contains a function relating the inputs to the output and
equations that give the `feasible answer region'. But, there are numerous
problems where we don't have a function that gives the input-output relationship
but yet we want to optimize the inputs for either maximizing or minimizing the
output based on some constraints on the inputs. For example, the problem of
socio-economic resource allocation in order to lessen viral infections.
\textbf{Stochastic programming} however, derives the \textit{expected objective
function} from historical data and then minimizes that, which is similar to
Machine Learning settings. {\textbf{Genetic algorithms}} or Evolutionary
Algorithms use search algorithms inspired by population behaviours that exist in
nature to solve optimization problems \cite{Kruse2022}. Just as in the previous
methods, the objective function needs to be fully defined. Genetic algorithms
truly shine when minimizing functions that are extremely complex, containing
many local minima/maxima, for example the Rosenbrock or Rastrigin function
\cite{test_functions}.

Another area that contains techniques that lead to decisions from data as
opposed to predictions from data is \textbf{statistical hypothesis testing}.
However, using hypothesis testing requires knowledge of the underlying
probability distributions and real world data does not explicitly express the
probability distribution it is generated from. \textbf{Data driven control}
provides a way to \textit{control} the input variables and thereby, optimize
them for a desired output however, requires knowledge of the input-output
relationships. When this relationship is unknown, \textbf{System identification}
is used to derive it. Data driven control and system identification are usually
applied to dynamic systems where the variables describing the system changes
with time. It is however possible to use another variable as a proxy for time to
hold information about how the system changes with that proxy variable. First,
we need to have a model structure, for example,

\begin{equation}
    y(k) + ay(k-1) = bu(k)
\end{equation}

Where \textbf{a} and \textbf{b} are adjustable (learnable) parameters. From this
model structure, we use historical (input, output) data to derive the learnable
parameters. Usually, the input and output data are recorded through time. If
we instead have static data, though it is possible to simply index every observation
and then use the indices as k, when the data is shuffled, it changes the dynamical 
system itself. Thus every shuffling order of the data gives a new dynamic system.
Besides, there is no inherent temporal information in the indices, they are simply
numbers with no physical or temporal meaning.

The above limitations called for new techniques that can optimize inputs simply
using historical data, by predicting the relationship between the inputs and
ouput and then using the predicted relationship to formulate an optimization
problem. \cite{Mandi_2024} proposed the name "\textbf{Predict, then optimize}"
for this framework and used the term `\textbf{Decision Focused Learning}' (DFL) to
describe the techniques. To our knowledge, \cite{spo} was the first to propose
such a method, wherein a trained predictive model was used to predict a cost
vector, given an input vector, and then the cost vector and input vector were
used to form an objective function for an optimizer to solve. The simplest
representation of such an objective function is,

\begin{equation}
    f(\mathbf{x}, \mathbf{c}) = \mathbf{x}.{\mathbf{c}}
\end{equation}

The predictive model needs to be trained in a supervised manner and so, we must
bear knowledge of how each variable in the input vector $\mathbf{x}$
individually affects the function $f$, which is not available in many real-world
scenarios. To our knowledge, \cite{Google_DFL}, was the first to apply DFL in a
real-world scenario to optimize the scheduling of live service calls in a
maternal and child health awareness program. They used socio-demographic
features (e.g., age, income, education) and historical engagement states (e.g.,
whether beneficiaries listened to calls) as inputs, while the engagement
outcomes (e.g., listening behavior) served as outputs. The model predicted
transition probabilities between engagement states (engaging/non-engaging) under
different actions (call/no call). These predicted probabilities were used to
compute Whittle Indices \cite{P.Whittle}, which prioritized beneficiaries for
live service calls based on their potential to benefit. 

Decision Focused Learning incorporates the decision making into the learning
algorithm (typically Backpropagation with Neural Networks). The function of the
decision, $f(\mathbf{x^*(\hat{c}))}$, is often non-differentiable and this poses
significant challenges in applying DFL \cite{Mandi_2024}. We can use a more
simplified approach wherein we are given input-output pairs, $\mathbf{(x,y)}$
and we have to produce a symbolic function, expressing the mapping between the
input and output. Thus, in the end, the problem is simply one of
\textbf{symbolic regression} (SR). After the symbolic expression of the function
is obtained via symbolic regression, we use it as the objective function.
Symbolic regression has been used to add a layer of `interpretability' to black
box machine learning methods that learn functions. We start with a set of
possible arithmetic operators known as the `library' and define the
dimensionality of the function space. For example, a library could be $L=\{
id(.), add(.,.), sub(.,.), mul(.,.), +1, -1\}$. This particular library can
compose the set of all polynomials in one variable with integer coefficients
\cite{SR_2024}. \cite{SR_2024} conducted a comprehensive survey on SR methods
and divided the techniques into 4 fundamental types; (1) Regression based
methods (2) Tree based methods (3) Physics inspired methods and (4) Mathematics
inspired methods. Genetic algorithms can actually be used as a tree based method
for SR. Recently, \textbf{Kolmogorov Arnold Networks} (KANs) were shown to
perform better at Symbolic Regression than any other method \cite{kans_2024}
\cite{yu2024kanmlpfairercomparison}. Kolmogorov Arnold networks have the same
skeleton as \textbf{Neural Networks} \cite{rumelhart1986learning}, but the
difference is KANs use learnable functions instead of real valued weights as the
edges of the computational graph. The vertices of the graph are simply summing
operators. The learnable activation functions is the main ingredient of KANs,
which are spline functions. This automatically produces the symbolic nature of
the learned function. In Neural Networks, this symbolic extraction is not as
easy. Thus, we propose using KANs to symbolically regress data and learn the
function that describes the relation between the inputs and output. Then using
the learned function as the objective function, solve the optimization problem.

In the following section. We discuss two approaches that we think could be adopted
to solve resource allocation problems based on historical data. 

\section{Formalization of the problem}
\label{sec: formalization}
Suppose the goal is to determine a combination of inputs $\{x\}_n$, constrained
by some conditions , that maximizes or minimizes an output, which is a function
of $\{x\}_n$. This problem can be described as,

% \begin{align}
%     \text{Given,}\\
%     f(\mathbf{x}) &= \mathbf{c.x},\\
%     g(\mathbf{x}) &\geq 0,\\

%     \text{Determine,}\\
%     \mathbf{x^*} &= \arg \min_x f(\mathbf{x})
% \end{align}

\begin{align}
    \text{Given,}  & \\
    f(\mathbf{x}) &= \frac{1}{2}.\mathbf{x}^T.H.\mathbf{x} + \mathbf{c^T.x}, \\
    g(\mathbf{x}) &\geq 0, \\
    h(\mathbf{x}) &= 0 \\
    \text{Determine,} & \\
    \mathbf{x}^* &= \arg \min_{\mathbf{x}} f(\mathbf{x}),
\end{align}

Where $\mathbf{c}$ is the coefficient vector that is unknown. Fortunately, ML
techniques provide a way to estimate this vector from data. If
$\hat{\mathbf{c}}$ is the predicted coefficient vector, the problem becomes,

\begin{align}
    \text{Given,}  & \\
    f(\mathbf{x}, \hat{\mathbf{c}}) &= \hat{\mathbf{c}} \cdot \mathbf{x}, \\
    g(\mathbf{x}) &\geq 0, \\
    h(\mathbf{x}) &= 0 \\
    \text{Determine,} & \\
    \mathbf{x}^*(\hat{\mathbf{c}}) &= \arg \min_{\mathbf{x}} f(\mathbf{x}, \hat{\mathbf{c}}),
\end{align}

Where $\mathbf{x}^*$ is a function of $\hat{\mathbf{c}}$ because it is dependent
on the prediction made by the learner. 

There are two ways of approaching this problem: (1) Modifying the ML learner so
that it incorporates the optimization problem into its learning algorithm or,
(2) Predicting and then, optimizing, which follows the method in \cite{spo}.

\subsection{Decision Focused learning (DFL)}
The core aspect of this approach is to incorporate the \textit{decision} derived
from the prediction, $\hat{\mathbf{c}}$,  into the learning algorithm. This is
done by quantifying the accuracy of the decision \textit{based on} the
prediction ($\mathbf{x}^*(\hat{\mathbf{c}})$), which we call \textit{Regret}.
The Regret is formally expressed below.

\begin{equation}
    Regret(\mathbf{x}^*(\hat{\mathbf{c}})) = f(\mathbf{x}^*(\hat{\mathbf{c}}), \mathbf{c}) - f(\mathbf{x}^*(\mathbf{c}), \mathbf{c})
\end{equation}

Where $f(\mathbf{x}^*(\mathbf{c}), \mathbf{c})$ is the decision that would have
been obtained had the optimizer exact knowledge of $\mathbf{c}$.
$f(\mathbf{x}^*(\mathbf{c}), \mathbf{c})$ is known as `full information
decision'. Here, as $\hat{\mathbf{c}} \rightarrow \mathbf{c}$, $Regret
\rightarrow 0$. So, the better the prediction, the lower the regret. Instead of
making a point estimation, it is also possible to estimate the distribution of
$\hat{\mathbf{c}}$ so that it is possible to guard against the worst case using
the distributional knowledge, however more challenging. We refer the reader to
\cite{Mandi_2024} for a more comprehensive overview of the learning algorithm.

\subsection{Express Symbolically, then optimize (ESO)}
In this method we follow a 2-stage procedure. In stage 1 we determine the
symbolic expression of the function that describes the data. In stage 2 we use
this function and the given constraints to determine the optimal combination of
inputs. Any symbolic regressor is viable. In the future, we shall experiment
with KANs and an ensemble of KANs in a \textbf{Mixture of Experts}
\cite{masoudnia_mixture_2014} manner to solve several optimization problems
simultaneously.

\section{Conclusions}
In this project, we discussed the techniques that are used to solve optimization
problems and how data driven methods can now adopt those techniques thanks to
symbolic regression. The approach is to first use symbolic regression and then
optimize. Besides analysing performance of different symbolic regressors, it is
important to study techniques to evaluate this 2 stage approach. Since, we are
essentially \textit{deriving} decisions from data, we imagine the type of
evaluation that is needed is post-hoc evaluation, after the decisions have been
prescribed and have been executed and further data has been collected on the
results. This is both a challenge and a novel way of evaluating predictive
models that affect downstream decisions. It brings into light that the
prediction may not be end of the workflow. 

\printbibliography
\end{document}